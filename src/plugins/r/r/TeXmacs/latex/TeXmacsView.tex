\Header{TeXmacsView}{function to start postscript TeXmacs interface}
\alias{end.view}{TeXmacsView}
\alias{start.view}{TeXmacsView}
\keyword{TeXmacs}{TeXmacsView}
\keyword{view}{TeXmacsView}
\keyword{device}{TeXmacsView}
\keyword{graphics}{TeXmacsView}
\begin{Description}\relax
Usually one uses the X11 device when interacting with TeXmacs. When
X11 is not available (for example in a remote login, start.view() will
open a postscript file with all defaults set correctly. end.view()
will close that file. While working, use v() to insert the current
file into the TeXmacs buffer.\end{Description}
\begin{Usage}
\begin{verbatim}
start.view()
end.view()
\end{verbatim}
\end{Usage}
\begin{Details}\relax
\end{Details}
\begin{Section}{Warning}
....\end{Section}
\begin{Author}\relax
Michael Lachmann Tamarlin\end{Author}
\begin{SeeAlso}\relax
See also \code{\Link{TeXmacs}},\end{SeeAlso}
\begin{Examples}
\begin{ExampleCode}
##---- Should be DIRECTLY executable !! ----
##-- ==>  Define data, use random,
##--    or do  help(data=index)  for the standard data sets.

## The function is currently defined as
function()
  {
    dev.off()
    op <- options("texmacs")$texmacs
    op$nox11 <- F
    options(texmacs=op)
    unlink(op$file)
  }
\end{ExampleCode}
\end{Examples}

